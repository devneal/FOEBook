\subsection{x86 and x86-64}
x86 CPUs have eight general-purpose registers. They are called \texttt{eax},
\texttt{ecx}, \texttt{edx}, \texttt{ebx}, \texttt{esp}, \texttt{ebp},
\texttt{esi}, \texttt{edi}. There are two other registers, \texttt{eip}, and
\texttt{eflags}, which have specific uses and cannot be written to directly.
Although each general-purpose register can technically be used for anything,
they are conventionally used for specific purposes.

\begin{itemize}
    \item \texttt{eax} (the accumulator) is used to store function return values 
    \item \texttt{esp} (the stack pointer) points to the top (lowest address) of
        the current stack frame 
    \item \texttt{ebp} (the base pointer) points to the base (highest) address
        of the current stack frame 
    \item \texttt{eip} (the instruction pointer) points to the next instruction
        that the CPU will execute. Each time an instruction is executed, the
        \texttt{eip} is set to the next instruction.
    \item \texttt{eflags} (the flags register) contains several single-bit flags
        that describe the state of the CPU
\end{itemize}

% TODO: add an image to demonstrate this
Some parts of each register can be manipulated independently of others. For
example, the lower 16 bits of \texttt{eax} are referred to as \texttt{ax}. The
lower 8 bits of \texttt{ax} are referred to as \texttt{al}, and the higher 8
bits of \texttt{ax} are referred to as \texttt{ah}. There is a similar naming
convention for \texttt{ecx}, \texttt{edx}, and \texttt{ebx}.

x86-64 extends the x86 registers mentioned above to 64 bits, and in doing so
replaces the `e' prefixes with `r' prefixes (i.e. \texttt{rax}, \texttt{rflags},
etc.). It also adds eight more general-purpose registers (\texttt{r8} through
\texttt{r15}), and eight 128-bit XMM registers.

% basic assembly
\subsection{Assembly, the Elven Tongue}
Although we write typically write programs in C, a CPU can only execute
instructions written in machine code. Machine code is unfortunately rather
difficult to for humans to read, so we instead use assembly, a language whose
instructions are one-to-one with machine code. Being comfortable reading
assembly will be invaluable while trying to understand and exploit programs, so
it will be useful to learn a few of the more common instructions.

\subsubsection{Intel vs. AT\&T}
Assembly can be written in one of two ways: intel syntax and at\&t syntax. Both
have the same instructions and convey the same information, but most people find
intel syntax a little bit easier to read. For the purposes of this book, all
assembly will be written in intel syntax. If you're ever unsure what syntax your
assembly is written in, just look for the \$ and \% characters that are heavily
used in at\&t syntax.

\subsubsection{Common Assembly Instructions}
Instructions in intel syntax are typically have one of two forms:
\texttt{<instruction> <destination> <source>} or \texttt{<instruction>
<argument>}. A few of the most common assembly instructions are listed below.

\begin{itemize}
    \item \texttt{mov <destination> <source>} - write data specified by source
        to destination
    \item \texttt{push <data>} - decrement the stack pointer, then write the
        specified data to the top of the stack
    \item \texttt{pop <data>} - write data at the top of the stack to argument,
        then increment the stack pointer
    \item \texttt{call <address>} - push the address of the next instruction,
        then move address into rip
    \item \texttt{ret} - move the address at the top of the stack into rip,
        then increment the stack pointer
    \item \texttt{nop} - do absolutely nothing
\end{itemize}

There are several assembly instructions to perform arithmetic and bitwise
operations on data.

\begin{itemize}
    \item \texttt{add <arg1> <arg2>} - writes arg1 + arg2 to arg1
    \item \texttt{sub <arg1> <arg2>} - writes arg1 - arg2 to arg1
    \item \texttt{imul <arg1> <arg2>} - writes arg1 * arg2 to arg1
    \item \texttt{idiv <arg>} - writes rax / arg to rdx:rax (or architecture
        equivalent)
    \item \texttt{xor <arg1> <arg2>} - writes arg1 $\wedge$ arg2 to arg1
    \item \texttt{and <arg1> <arg2>} - writes arg1 \& arg2 to arg1
\end{itemize}

\begin{aside}
    $\wedge$ denotes bitwise xor, a binary operator which is 1 only when its
    arguments are different. \& denotes bitwise and, another binary operator
    which is 1 only when both its arguments are 1.
\end{aside}

Finally, there are a family of jump instructions that deserve special attention.
The \texttt{jmp} instruction simply redirects execution to the address specified
by its argument. Each of the others checks \texttt{rflags} and will only
redirect execution if the flags meet a certain condition.

\begin{itemize}
    \item \texttt{jmp} - unconditional jump
    \item \texttt{je} - jump if equal (to zero)
    \item \texttt{jne} - jump if not equal (to zero)
    \item \texttt{jl} - jump if less (than zero)
    \item \texttt{jle} - jump if less than or equal (to zero)
    \item \texttt{jg} - jump if greater (than zero)
    \item \texttt{jge} - jump if greater than or equal (to zero)
    \item \texttt{cmp} - perform subtraction, but ignore the result (only set
        \texttt{rflags})
    \item \texttt{test} - perform and, but ignore the result (only set
        \texttt{rflags})
\end{itemize}

When an assembly instruction references memory, it must specify both the
location of size of that memory. The intel syntax for addressing memory is
\texttt{<size> PTR [<addr>]}, where the size is one of the following:
\begin{itemize}
    \item \texttt{BYTE} - 1 byte
    \item \texttt{WORD} - 2 bytes
    \item \texttt{DWORD} - 4 bytes
    \item \texttt{QWORD} - 8 bytes
\end{itemize}

A snippet of assembly code is below. You now know enough to figure out what it
does.

\begin{lstlisting}
40052e:  mov    DWORD PTR [rbp-0x8],0x0
400535:  mov    DWORD PTR [rbp-0x4],0x0
40053c:  jmp    400548
40053e:  mov    eax,DWORD PTR [rbp-0x4]
400541:  add    DWORD PTR [rbp-0x8],eax
400544:  add    DWORD PTR [rbp-0x4],0x1
400548:  cmp    DWORD PTR [rbp-0x4],0xa
40054c:  jle    40053e
\end{lstlisting}

The key to reading assembly is to divide the code into sections and detail
exactly what each line is doing. Then, you can determine how each of the
sections interact with each other to learn what the entire program is doing.

\begin{lstlisting}
40052e:  mov    DWORD PTR [rbp-0x8],0x0
400535:  mov    DWORD PTR [rbp-0x4],0x0
40053c:  jmp    400548
\end{lstlisting}

First the program writes 0 to \texttt{rbp-0x8} and \texttt{rbp-0x4}, then jumps
to \texttt{0x400548}.

\begin{lstlisting}
400548:  cmp    DWORD PTR [rbp-0x4],0xa
40054c:  jle    40053e
\end{lstlisting}

After the jump, the program compares \texttt{rbp-0x4} and \texttt{0xa}. More
specifically, it computes \texttt{[rbp-0x4] - 0xa}, sets \texttt{rflags}
according to the result, then throws it away. It then jumps to \texttt{0x40053e}
if \texttt{rflags} is set according to a value less than or equal to zero.
Since \texttt{rbp-0x4} was just set to 0 the result of the subtraction is
\texttt{-0xa}, so the jump is taken.

\begin{lstlisting}
40053e:  mov    eax,DWORD PTR [rbp-0x4]
400541:  add    DWORD PTR [rbp-0x8],eax
400544:  add    DWORD PTR [rbp-0x4],0x1
\end{lstlisting}

The program moves \texttt{[rbp-0x4]} into \texttt{eax}, computes
\texttt{[rbp-0x8] + eax}, and overwrites \texttt{[rbp-0x8]} with the result. At
this point, but \texttt{[rbp-0x8]} and \texttt{[rbp-0x4]} are 0, so the only
effect is setting \texttt{[rbp-0x4]} to 1. After this block execution continues
at \texttt{0x400548}, which we have alredy analyzed.

We can now analyze these sections of code. There are two variables, one at
\texttt{[rbp-0x4]} and the other at \texttt{[rbp-0x8]}. The one at
\texttt{[rbp-0x4]} is only changed by incrementing and used to check when to
stop looping, so it functions as a counter. The variable at \texttt{[rbp-0x8]}
stores a running sum. The instructions from \texttt{0x40053e} to
\texttt{0x40054c} function as a loop, which runs 11 times (for values 0-10).
Each time the loop runs it first adds the counter to the running sum then
increments the counter. Therefore this program is calculating the sum of the
first 10 integers (including zero), leaving the result in \texttt{[rbp-0x8]}.

This is commonly used with the \texttt{mov} instruction, i.e. \texttt{mov
QWORD PTR [rbp-0x8],rax}. 

% variables, stack frames, heap basics
% TODO: include stack frame diagram
\subsection{The Stack and Heap}
The most important use of the stack is in handling nested function calls. In
order to make this work seamlessly, the functions follow a calling convention
which outlines instructions for both the calling function (the caller) and the
called function (the callee). The calling convention is as follows:

The caller shall:
\begin{enumerate}
    \item Prepare the callee's arguments by either loading them into registers
        (x86-64) or pushing them onto the stack in reverse order (x86)
    \item Execute the \texttt{call} instruction to jump to the new function and
        push the address of the next instruction onto the stack
    \item After the callee returns, clear the stack of any callee arguments
\end{enumerate}

At the \emph{start} of execution, the callee shall:
\begin{enumerate}
    \item Push the caller's base pointer onto the stack
    \item Move the base pointer to point to the caller's saved base pointer
    \item Subtract from the base pointer to make room for any local variables
\end{enumerate}

At the \emph{end} of execution, the callee shall:
\begin{enumerate}
    \item Leave the return value in the accumulator
    \item Move the stack pointer to point to the caller's saved base pointer
    \item Restore the caller's base pointer by popping if off of the stack
    \item Execute the \texttt{ret} instruction to return control to the caller
\end{enumerate}

Note that the callee essentially undoes everything it did to build its new stack
frame after it finishes execution. This way the caller can continue execution
after finishing the calling convention with it's stack frame intact.
Additionally, this calling convention allows for the callee to call other
functions during it's execution, since the stack frames they build will be
popped off the stack after they terminate. This means that we can nest function
calls indefinitely as long as there is room on the stack to keep building stack
frames!

You now know enough to understand a basic program written in assembly. Take this
one, for example.

\begin{lstlisting}
// elf.c 
#include <stdio.h>

int main(void) {
    int num;
    printf("ELF example\n");
    scanf("%d\n", &num);
    return 0;
}
\end{lstlisting}

If you to compile this program with \texttt{gcc -o elf elf.c}, you will create
a new ELF file called \texttt{elf}.

\begin{lstlisting}
> gcc -o elf elf.c 
> ls -l elf
-rwxrwxr-x 1 devneal devneal 8720 Nov  9 11:28 elf
> 
\end{lstlisting}

We can use a tool called \texttt{objdump} to read a compiled program's assembly.
Run \texttt{objdump -M intel -d elf} to see the disassembled program's machine
code.

\begin{lstlisting}
00000000004005f6 <main>:
  4005f6:       55                      push   rbp
  4005f7:       48 89 e5                mov    rbp,rsp
  4005fa:       48 83 ec 10             sub    rsp,0x10
  4005fe:       64 48 8b 04 25 28 00    mov    rax,QWORD PTR fs:0x28
  400605:       00 00 
  400607:       48 89 45 f8             mov    QWORD PTR [rbp-0x8],rax
  40060b:       31 c0                   xor    eax,eax
  40060d:       bf d4 06 40 00          mov    edi,0x4006d4
  400612:       e8 99 fe ff ff          call   4004b0 <puts@plt>
  400617:       48 8d 45 f4             lea    rax,[rbp-0xc]
  40061b:       48 89 c6                mov    rsi,rax
  40061e:       bf e0 06 40 00          mov    edi,0x4006e0
  400623:       b8 00 00 00 00          mov    eax,0x0
  400628:       e8 b3 fe ff ff          call   4004e0 <__isoc99_scanf@plt>
  40062d:       b8 00 00 00 00          mov    eax,0x0
  400632:       48 8b 55 f8             mov    rdx,QWORD PTR [rbp-0x8]
  400636:       64 48 33 14 25 28 00    xor    rdx,QWORD PTR fs:0x28
  40063d:       00 00 
  40063f:       74 05                   je     400646 <main+0x50>
  400641:       e8 7a fe ff ff          call   4004c0 <__stack_chk_fail@plt>
  400646:       c9                      leave  
  400647:       c3                      ret    
  400648:       0f 1f 84 00 00 00 00    nop    DWORD PTR [rax+rax*1+0x0]
  40064f:       00
\end{lstlisting}

The first three instructions are the function prologue, creating a new stack
frame.
\begin{lstlisting}
4005f6:       55                      push   rbp
4005f7:       48 89 e5                mov    rbp,rsp
4005fa:       48 83 ec 10             sub    rsp,0x10
\end{lstlisting}

The next two instructions may seem strange. The program reads a \texttt{QWORD}
from somewhere into \texttt{rax}, then stores that values on the stack at
\texttt{rbp-0x8}. It then uses a clever trick to zero out \texttt{eax}.
\begin{lstlisting}
4005fe:       64 48 8b 04 25 28 00    mov    rax,QWORD PTR fs:0x28
400605:       00 00 
400607:       48 89 45 f8             mov    QWORD PTR [rbp-0x8],rax
40060b:       31 c0                   xor    eax,eax
\end{lstlisting}

Next is a call to \texttt{puts()}. We can see the first argument (presumably a
format string) being moved into \texttt{edi} preceding the call.
\begin{lstlisting}
40060d:       bf d4 06 40 00          mov    edi,0x4006d4
400612:       e8 99 fe ff ff          call   4004b0 <puts@plt>
\end{lstlisting}

Now there's a call to \texttt{scanf()}. Since we called \texttt{scanf()} with
two arguments, but \texttt{rdi} and \texttt{rsi} are set before the call. It
then moves \texttt{0x0} into \texttt{eax} in order to return 0.

\begin{lstlisting}
400617:       48 8d 45 f4             lea    rax,[rbp-0xc]
40061b:       48 89 c6                mov    rsi,rax
40061e:       bf e0 06 40 00          mov    edi,0x4006e0
400623:       b8 00 00 00 00          mov    eax,0x0
400628:       e8 b3 fe ff ff          call   4004e0 <__isoc99_scanf@plt>
40062d:       b8 00 00 00 00          mov    eax,0x0
\end{lstlisting}

This is followed by a few more instructions involving the mysterious value at
\texttt{rbp-0x8}. Their purpose can be ignored for now, but we can tell that the
program is comparing the value on the stack to the one that was originally
placed there.
\begin{lstlisting}
400632:       48 8b 55 f8             mov    rdx,QWORD PTR [rbp-0x8]
400636:       64 48 33 14 25 28 00    xor    rdx,QWORD PTR fs:0x28
40063d:       00 00 
40063f:       74 05                   je     400646 <main+0x50>
400641:       e8 7a fe ff ff          call   4004c0 <__stack_chk_fail@plt>
\end{lstlisting}

Last, the program exits by executing the function epilogue followed by the
\texttt{ret} instruction.
\begin{lstlisting}
400646:       c9                      leave  
400647:       c3                      ret    
\end{lstlisting}

% memory segmentation, readelf, objdump
% TODO: add diagram
\subsection{Memory Layout}
Memory in a running program can be divided into sections, each of which is used
for a specific purpose. They are, in order from lower addresses to higher
addresses, \texttt{.text}, \texttt{.data}, \texttt{.bss}, \texttt{heap}, and
\texttt{stack}.

\begin{itemize}
    \item The \texttt{.text} section stores the program's executable code and is
never writable.
    \item The \texttt{.data} section stores any static or global variables (in C
terminology) that are initialized in the source code and writable.
    \item The \texttt{.bss} secition stores any static or global variables that
are initialized to zero or not explicitly initialized in the source code.
    \item The \texttt{heap} is a section of memory which can be dynamically
allocated at runtime. The heap grows downward, toward higher memory addresses.
    \item The \texttt{stack} is a section of memory which is used to store
local variables and handle nested function calls. The stack grows upward, toward
lower memory addresses.
\end{itemize}

% symbols, sections, and segments
\subsection{ELF Anatomy}
ELF, or Executable and Linkable Format, is the most common type of executable
for Linux systems. Whenever you compile a program with `gcc`, the result is and
ELF binary.

\begin{lstlisting}
> file elf
elf: ELF 64-bit LSB executable, x86-64, version 1 (SYSV), dynamically linked, interpreter /lib64/ld-linux-x86-64.so.2, for GNU/Linux 2.6.32, BuildID[sha1]=6dc45433a562bb0eb99f962510ad71b3da43095d, not stripped
> 
\end{lstlisting}

The output of the \texttt{file} command indicates that this ELF binary was
compiled for a litte-endian (Least Significant Byte) x86-64 architecture. We can
see more information with the \texttt{readelf} command.

\begin{lstlisting}
> readelf --file-header elf
ELF Header:
  Magic:   7f 45 4c 46 02 01 01 00 00 00 00 00 00 00 00 00 
  Class:                             ELF64
  Data:                              2's complement, little endian
  Version:                           1 (current)
  OS/ABI:                            UNIX - System V
  ABI Version:                       0
  Type:                              EXEC (Executable file)
  Machine:                           Advanced Micro Devices X86-64
  Version:                           0x1
  Entry point address:               0x400430
  Start of program headers:          64 (bytes into file)
  Start of section headers:          6616 (bytes into file)
  Flags:                             0x0
  Size of this header:               64 (bytes)
  Size of program headers:           56 (bytes)
  Number of program headers:         9
  Size of section headers:           64 (bytes)
  Number of section headers:         31
  Section header string table index: 28
> 
\end{lstlisting}

We don't need most of this information right now, but there are a few
interesting things. The "Magic" field indicates the first few bytes in the file,
which always starts with \texttt{7f} followed by the ascii representation of the
characters `E' `L' `F'. The "Class" field is \texttt{ELF64}, indicating that
this executable was compiled for a 64-bit architecture, and the "Data" field
shows that the executable uses little-endian byte order. \texttt{readelf} is a
useful tool for retreiving information about binaries, so it's worth getting
familiar with it.

\subsubsection{Symbols, Sections, and Segments}
ELF binaries can be organized into \emph{symbols}, \emph{sections}, and
\emph{segments}. This grouping is heirarchical: segments are groups of sections
and each section contains several symbols. Symbols are simply names for memory
locations. Each symbol is identified by it's location in memory and it's size.
We can view an ELF file's symbols by passing the \texttt{--symbols} flag to
\texttt{readelf}.

\begin{lstlisting}
> readelf --symbols elf | tail -n 10
    57: 00000000004005c0     4 OBJECT  GLOBAL DEFAULT   16 _IO_stdin_used
    58: 0000000000400540   101 FUNC    GLOBAL DEFAULT   14 __libc_csu_init
    59: 0000000000601040     0 NOTYPE  GLOBAL DEFAULT   26 _end
    60: 0000000000400430    42 FUNC    GLOBAL DEFAULT   14 _start
    61: 0000000000601038     0 NOTYPE  GLOBAL DEFAULT   26 __bss_start
    62: 0000000000400526    21 FUNC    GLOBAL DEFAULT   14 main
    63: 0000000000000000     0 NOTYPE  WEAK   DEFAULT  UND _Jv_RegisterClasses
    64: 0000000000601038     0 OBJECT  GLOBAL HIDDEN    25 __TMC_END__
    65: 0000000000000000     0 NOTYPE  WEAK   DEFAULT  UND _ITM_registerTMCloneTable
    66: 00000000004003c8     0 FUNC    GLOBAL DEFAULT   11 _init
> 
\end{lstlisting}

Here we can see that the symbol for \texttt{main()} is located at address
\texttt{0x400526} and has a size of 21 bytes. The compiler adds many more
symbols for the linker to use.

Each section of the binary is used for a different purpose. We've already seen
the  \texttt{.text} section, which stores machine code, the \texttt{.data}
section, which stores initialized global or static variables, and the
\texttt{.bss} section, which stores uninitialized global or static variables. To
list all of the sections in an ELF binary, pass the \texttt{--sections} flag to
\texttt{readelf}.

\begin{lstlisting}
> readelf --sections elf
There are 31 section headers, starting at offset 0x19d8:

Section Headers:
  [Nr] Name              Type             Address           Offset
       Size              EntSize          Flags  Link  Info  Align
  [ 0]                   NULL             0000000000000000  00000000
       0000000000000000  0000000000000000           0     0     0
  [ 1] .interp           PROGBITS         0000000000400238  00000238
       000000000000001c  0000000000000000   A       0     0     1
  [ 2] .note.ABI-tag     NOTE             0000000000400254  00000254
       0000000000000020  0000000000000000   A       0     0     4
  [ 3] .note.gnu.build-i NOTE             0000000000400274  00000274
       0000000000000024  0000000000000000   A       0     0     4
  [ 4] .gnu.hash         GNU_HASH         0000000000400298  00000298
       000000000000001c  0000000000000000   A       5     0     8
  [ 5] .dynsym           DYNSYM           00000000004002b8  000002b8
       0000000000000060  0000000000000018   A       6     1     8
  [ 6] .dynstr           STRTAB           0000000000400318  00000318
       000000000000003d  0000000000000000   A       0     0     1
  [ 7] .gnu.version      VERSYM           0000000000400356  00000356
       0000000000000008  0000000000000002   A       5     0     2
  [ 8] .gnu.version_r    VERNEED          0000000000400360  00000360
       0000000000000020  0000000000000000   A       6     1     8
  [ 9] .rela.dyn         RELA             0000000000400380  00000380
       0000000000000018  0000000000000018   A       5     0     8
  [10] .rela.plt         RELA             0000000000400398  00000398
       0000000000000030  0000000000000018  AI       5    24     8
  [11] .init             PROGBITS         00000000004003c8  000003c8
       000000000000001a  0000000000000000  AX       0     0     4
  [12] .plt              PROGBITS         00000000004003f0  000003f0
       0000000000000030  0000000000000010  AX       0     0     16
  [13] .plt.got          PROGBITS         0000000000400420  00000420
       0000000000000008  0000000000000000  AX       0     0     8
  [14] .text             PROGBITS         0000000000400430  00000430
       0000000000000182  0000000000000000  AX       0     0     16
  [15] .fini             PROGBITS         00000000004005b4  000005b4
       0000000000000009  0000000000000000  AX       0     0     4
  [16] .rodata           PROGBITS         00000000004005c0  000005c0
       0000000000000010  0000000000000000   A       0     0     4
  [17] .eh_frame_hdr     PROGBITS         00000000004005d0  000005d0
       0000000000000034  0000000000000000   A       0     0     4
  [18] .eh_frame         PROGBITS         0000000000400608  00000608
       00000000000000f4  0000000000000000   A       0     0     8
  [19] .init_array       INIT_ARRAY       0000000000600e10  00000e10
       0000000000000008  0000000000000000  WA       0     0     8
  [20] .fini_array       FINI_ARRAY       0000000000600e18  00000e18
       0000000000000008  0000000000000000  WA       0     0     8
  [21] .jcr              PROGBITS         0000000000600e20  00000e20
       0000000000000008  0000000000000000  WA       0     0     8
  [22] .dynamic          DYNAMIC          0000000000600e28  00000e28
       00000000000001d0  0000000000000010  WA       6     0     8
  [23] .got              PROGBITS         0000000000600ff8  00000ff8
       0000000000000008  0000000000000008  WA       0     0     8
  [24] .got.plt          PROGBITS         0000000000601000  00001000
       0000000000000028  0000000000000008  WA       0     0     8
  [25] .data             PROGBITS         0000000000601028  00001028
       0000000000000010  0000000000000000  WA       0     0     8
  [26] .bss              NOBITS           0000000000601038  00001038
       0000000000000008  0000000000000000  WA       0     0     1
  [27] .comment          PROGBITS         0000000000000000  00001038
       0000000000000034  0000000000000001  MS       0     0     1
  [28] .shstrtab         STRTAB           0000000000000000  000018ca
       000000000000010c  0000000000000000           0     0     1
  [29] .symtab           SYMTAB           0000000000000000  00001070
       0000000000000648  0000000000000018          30    47     8
  [30] .strtab           STRTAB           0000000000000000  000016b8
       0000000000000212  0000000000000000           0     0     1
Key to Flags:
  W (write), A (alloc), X (execute), M (merge), S (strings), l (large)
  I (info), L (link order), G (group), T (TLS), E (exclude), x (unknown)
  O (extra OS processing required) o (OS specific), p (processor specific)
> 
\end{lstlisting}

We can see from the output above that the \texttt{.text} section is not writable
(as expected), but the \texttt{.data} and \texttt{.bss} sections are. Among the
new sections are \texttt{.plt} and \texttt{.got.plt}, both of which are
important for linking.

We can view the binary's segments with \texttt{readelf --segments}.

\begin{lstlisting}
> readelf --segments elf

Elf file type is EXEC (Executable file)
Entry point 0x400430
There are 9 program headers, starting at offset 64

Program Headers:
  Type           Offset             VirtAddr           PhysAddr
                 FileSiz            MemSiz              Flags  Align
  PHDR           0x0000000000000040 0x0000000000400040 0x0000000000400040
                 0x00000000000001f8 0x00000000000001f8  R E    8
  INTERP         0x0000000000000238 0x0000000000400238 0x0000000000400238
                 0x000000000000001c 0x000000000000001c  R      1
      [Requesting program interpreter: /lib64/ld-linux-x86-64.so.2]
  LOAD           0x0000000000000000 0x0000000000400000 0x0000000000400000
                 0x00000000000006fc 0x00000000000006fc  R E    200000
  LOAD           0x0000000000000e10 0x0000000000600e10 0x0000000000600e10
                 0x0000000000000228 0x0000000000000230  RW     200000
  DYNAMIC        0x0000000000000e28 0x0000000000600e28 0x0000000000600e28
                 0x00000000000001d0 0x00000000000001d0  RW     8
  NOTE           0x0000000000000254 0x0000000000400254 0x0000000000400254
                 0x0000000000000044 0x0000000000000044  R      4
  GNU_EH_FRAME   0x00000000000005d0 0x00000000004005d0 0x00000000004005d0
                 0x0000000000000034 0x0000000000000034  R      4
  GNU_STACK      0x0000000000000000 0x0000000000000000 0x0000000000000000
                 0x0000000000000000 0x0000000000000000  RW     10
  GNU_RELRO      0x0000000000000e10 0x0000000000600e10 0x0000000000600e10
                 0x00000000000001f0 0x00000000000001f0  R      1

 Section to Segment mapping:
  Segment Sections...
   00     
   01     .interp 
   02     .interp .note.ABI-tag .note.gnu.build-id .gnu.hash .dynsym .dynstr .gnu.version .gnu.version_r .rela.dyn .rela.plt .init .plt .plt.got .text .fini .rodata .eh_frame_hdr .eh_frame 
   03     .init_array .fini_array .jcr .dynamic .got .got.plt .data .bss 
   04     .dynamic 
   05     .note.ABI-tag .note.gnu.build-id 
   06     .eh_frame_hdr 
   07     
   08     .init_array .fini_array .jcr .dynamic .got 
> 
\end{lstlisting}

This shows us the segments, their permissions, and which sections are contained
in each segment. For example, the \texttt{.data} and \texttt{.bss} sections are
located in the third segment, which has a type of \texttt{LOAD}. We also see
that this segment has both read and write permissions.

% Shared libraries
\subsubsection{PLT and GOT}
One of the most useful attributes of ELF binaries is the fact that they can
use each other's data through a process called linking. Although linking is
conceptually simple, it's implementation is rather complex due to the fact that
shared libraries must function properly regardless of where they are loaded into
memory. This means that ELF binaries need some way to determine the locations of
their shared library functions at runtime. ELF binaries use two data structures
to achieve this - the Procedure Linkage Table (PLT) and Global Offset Table
(GOT).

The PLT is a list of code stubs which are called in place of shared library
functions, and the GOT is a list of pointers where the PLT will redirect
execution. Each shared library function in the ELF has an entry in both the PLT
and the GOT. The first entry in the PLT is used to call the resolver, and each
following entry is used to call a shared library function. Each PLT entry other
than the first consists of a jump to the corresponding address in the GOT, a
push onto the stack to prepare the resolver, and a jump to the resolver. When
the program is first loaded, each shared function's GOT entry points back to the
PLT instructions to prepare and call the resolver. When the function is first
called, the resolver will find the address of the function in libc (or other
library), write the address to the GOT, and call the function. The next time the
function is called, the PLT will redirect execution to the library code, so the
resolution is only performed once.

% gdb
\subsection{Stepping through with GDB}
We can use a debugger to step through an ELF file's execution one instruction at
a time, inspecting and modifying its data as we please. This is a powerful tool
for learning about a new executable. We're going to use the GNU debugger (GDB),
since it's widely available and very powerful. To start debugging a program, run
gdb <program>. Not much will happen, since gdb is run only through the command
line.

\begin{lstlisting}
> gdb elf
GNU gdb (Ubuntu 7.11.1-0ubuntu1~16.5) 7.11.1
Copyright (C) 2016 Free Software Foundation, Inc.
License GPLv3+: GNU GPL version 3 or later <http://gnu.org/licenses/gpl.html>
This is free software: you are free to change and redistribute it.
There is NO WARRANTY, to the extent permitted by law.  Type "show copying"
and "show warranty" for details.
This GDB was configured as "x86_64-linux-gnu".
Type "show configuration" for configuration details.
For bug reporting instructions, please see:
<http://www.gnu.org/software/gdb/bugs/>.
Find the GDB manual and other documentation resources online at:
<http://www.gnu.org/software/gdb/documentation/>.
For help, type "help".
Type "apropos word" to search for commands related to "word"...
Reading symbols from elf...(no debugging symbols found)...done.
(gdb) 
\end{lstlisting}

It's useful to have a few survival gdb commands to get started. These can get
you pretty far:

\begin{itemize}
    \item help - get information on how to use a command
    \item disassemble - show disassembly of a function
    \item break - set a breakpoint
    \item run - run the program from the beginning
    \item where - display your current location
    \item info registers - display register status
    \item info breakpoints - display breakpoint status
    \item x - examine memory
    \item display - display memory at each breakpoint
    \item nexti - execute an instruction without following jumps / calls
    \item stepi - execute an instruction following jumps / calls
    \item continue - resume execution from a breakpoint
\end{itemize}

As explained in the welcome message, you can use \texttt{help} or
\texttt{apropos} to get information if you get lost.

If you type part of a command and press tab twice, \texttt{gdb} will suggest
ways to finish the command. If there is only one way to complete the command you
\emph{could} press tab to finish the command, but \texttt{gdb} will actually
execute the completed command automatically. This walkthrough will use the full
commands so that you can see them, but as you use the commands more, you'll want
to start using the abbreviated versions.

To start, we can view the disassembly of \texttt{main()} by running
\texttt{disassemble main}. However, by default this will display the assembly in
att syntax. To switch to intel syntax, run \texttt{set disassembly intel}.

\begin{lstlisting}
(gdb) set disassembly-flavor intel
(gdb) disassemble main
Dump of assembler code for function main:
   0x00000000004005f6 <+0>:	push   rbp
   0x00000000004005f7 <+1>:	mov    rbp,rsp
   0x00000000004005fa <+4>:	sub    rsp,0x10
   0x00000000004005fe <+8>:	mov    rax,QWORD PTR fs:0x28
   0x0000000000400607 <+17>:	mov    QWORD PTR [rbp-0x8],rax
   0x000000000040060b <+21>:	xor    eax,eax
   0x000000000040060d <+23>:	mov    edi,0x4006d4
   0x0000000000400612 <+28>:	call   0x4004b0 <puts@plt>
   0x0000000000400617 <+33>:	lea    rax,[rbp-0xc]
   0x000000000040061b <+37>:	mov    rsi,rax
   0x000000000040061e <+40>:	mov    edi,0x4006e0
   0x0000000000400623 <+45>:	mov    eax,0x0
   0x0000000000400628 <+50>:	call   0x4004e0 <__isoc99_scanf@plt>
   0x000000000040062d <+55>:	mov    eax,0x0
   0x0000000000400632 <+60>:	mov    rdx,QWORD PTR [rbp-0x8]
   0x0000000000400636 <+64>:	xor    rdx,QWORD PTR fs:0x28
   0x000000000040063f <+73>:	je     0x400646 <main+80>
   0x0000000000400641 <+75>:	call   0x4004c0 <__stack_chk_fail@plt>
   0x0000000000400646 <+80>:	leave  
   0x0000000000400647 <+81>:	ret    
End of assembler dump.
(gdb) 
\end{lstlisting}

To pause execution at the start of \texttt{main()}, we'll first set a breakpoint
there, then run the program.

\begin{lstlisting}
(gdb) break main
Breakpoint 1 at 0x4005fa
(gdb) run
Starting program: /home/devneal/Security/REFE/textbook/elf 

Breakpoint 1, 0x00000000004005fa in main ()
(gdb) 
\end{lstlisting}

From here we can verify our location with the \texttt{where} and \texttt{info
registers} commands. Since we only need to see the location of \texttt{rip}, we
can use \texttt{info register rip} to see it exclusively.

\begin{lstlisting}
(gdb) where
#0  0x00000000004005fa in main ()
(gdb) info register rip
rip            0x4005fa	0x4005fa <main+4>
(gdb) 
\end{lstlisting}

From here we can use the \texttt{x} command to examine the state of the program.
\texttt{x/5i \$rip} will display the next 5 instructions to be executed, and
\texttt{x/8xw \$rsp} will display the first 5 hexadecimal words on the top of the
stack. You can get more information on how to use \texttt{x} with \texttt{help
x}.

\begin{lstlisting}
(gdb) x/5i $rip
=> 0x4005fa <main+4>:	sub    rsp,0x10
   0x4005fe <main+8>:	mov    rax,QWORD PTR fs:0x28
   0x400607 <main+17>:	mov    QWORD PTR [rbp-0x8],rax
   0x40060b <main+21>:	xor    eax,eax
   0x40060d <main+23>:	mov    edi,0x4006d4
(gdb) x/8xw $rsp
0x7fffffffdee0:	0x00400650	0x00000000	0xf7a2d830	0x00007fff
0x7fffffffdef0:	0x00000000	0x00000000	0xffffdfc8	0x00007fff
(gdb) 
\end{lstlisting}

From the output above, we can see that the next instruction will subtract
\texttt{0x10} from \texttt{\$rsp}. We can execute this instruction by running
\texttt{nexti} and verify that it behaved as expected.

\begin{lstlisting}
(gdb) nexti
0x00000000004005fe in main ()
(gdb) x/5i $rip
=> 0x4005fe <main+8>:	mov    rax,QWORD PTR fs:0x28
   0x400607 <main+17>:	mov    QWORD PTR [rbp-0x8],rax
   0x40060b <main+21>:	xor    eax,eax
   0x40060d <main+23>:	mov    edi,0x4006d4
   0x400612 <main+28>:	call   0x4004b0 <puts@plt>
(gdb) x/8xw $rsp
0x7fffffffded0:	0xffffdfc0	0x00007fff	0x00000000	0x00000000
0x7fffffffdee0:	0x00400650	0x00000000	0xf7a2d830	0x00007fff
(gdb) 
\end{lstlisting}

As expected, \texttt{rip} is now pointing at the next instruction and
\texttt{rsp} has been decremented by \texttt{0x10} (4 words). We can use the
\texttt{display} command to view \texttt{rip} and \texttt{rsp} every time
execution stops.

\begin{lstlisting}
(gdb) display/5i $rip
1: x/5i $rip
=> 0x4005fe <main+8>:	mov    rax,QWORD PTR fs:0x28
   0x400607 <main+17>:	mov    QWORD PTR [rbp-0x8],rax
   0x40060b <main+21>:	xor    eax,eax
   0x40060d <main+23>:	mov    edi,0x4006d4
   0x400612 <main+28>:	call   0x4004b0 <puts@plt>
(gdb) display/5xw $rsp
2: x/5xw $rsp
0x7fffffffded0:	0xffffdfc0	0x00007fff	0x00000000	0x00000000
0x7fffffffdee0:	0x00400650
(gdb) 
\end{lstlisting}

Use the \texttt{nexti} command to step through a few more instructions, and
the stack and instruction pointers will update automatically.

Next we'll set a breakpoint at the call to \texttt{scanf()}. We can find
location of the \texttt{call} instruction with \texttt{disassemble}, set a
breakpoint there with \texttt{break}, and stop at it with \texttt{continue}.

\begin{lstlisting}
(gdb) disassemble main
Dump of assembler code for function main:
   0x00000000004005f6 <+0>:	push   rbp
   0x00000000004005f7 <+1>:	mov    rbp,rsp
   0x00000000004005fa <+4>:	sub    rsp,0x10
   0x00000000004005fe <+8>:	mov    rax,QWORD PTR fs:0x28
   0x0000000000400607 <+17>:	mov    QWORD PTR [rbp-0x8],rax
   0x000000000040060b <+21>:	xor    eax,eax
   0x000000000040060d <+23>:	mov    edi,0x4006d4
   0x0000000000400612 <+28>:	call   0x4004b0 <puts@plt>
   0x0000000000400617 <+33>:	lea    rax,[rbp-0xc]
   0x000000000040061b <+37>:	mov    rsi,rax
   0x000000000040061e <+40>:	mov    edi,0x4006e0
   0x0000000000400623 <+45>:	mov    eax,0x0
   0x0000000000400628 <+50>:	call   0x4004e0 <__isoc99_scanf@plt>
   0x000000000040062d <+55>:	mov    eax,0x0
   0x0000000000400632 <+60>:	mov    rdx,QWORD PTR [rbp-0x8]
   0x0000000000400636 <+64>:	xor    rdx,QWORD PTR fs:0x28
   0x000000000040063f <+73>:	je     0x400646 <main+80>
   0x0000000000400641 <+75>:	call   0x4004c0 <__stack_chk_fail@plt>
   0x0000000000400646 <+80>:	leave  
   0x0000000000400647 <+81>:	ret    
End of assembler dump.
(gdb) break *0x400628
Breakpoint 2 at 0x400628
(gdb) run
Starting program: /home/devneal/Security/REFE/textbook/elf 

Breakpoint 1, 0x00000000004005fa in main ()
1: x/5i $rip
=> 0x4005fa <main+4>:	sub    rsp,0x10
   0x4005fe <main+8>:	mov    rax,QWORD PTR fs:0x28
   0x400607 <main+17>:	mov    QWORD PTR [rbp-0x8],rax
   0x40060b <main+21>:	xor    eax,eax
   0x40060d <main+23>:	mov    edi,0x4006d4
2: x/5xw $rsp
0x7fffffffdee0:	0x00400650	0x00000000	0xf7a2d830	0x00007fff
0x7fffffffdef0:	0x00000000
(gdb) continue 
Continuing.
ELF example

Breakpoint 2, 0x0000000000400628 in main ()
1: x/5i $rip
=> 0x400628 <main+50>:	call   0x4004e0 <__isoc99_scanf@plt>
   0x40062d <main+55>:	mov    eax,0x0
   0x400632 <main+60>:	mov    rdx,QWORD PTR [rbp-0x8]
   0x400636 <main+64>:	xor    rdx,QWORD PTR fs:0x28
   0x40063f <main+73>:	je     0x400646 <main+80>
2: x/5xw $rsp
0x7fffffffded0:	0xffffdfc0	0x00007fff	0x3f318f00	0xf8ca2299
0x7fffffffdee0:	0x00400650
(gdb) 
\end{lstlisting}

Now we can examine the arguments to \texttt{scanf()}. The first is a format
string, which can be read by passing the \texttt{/s} flag to \texttt{x}, and the
second is the address on the stack where the input will be stored.

\begin{lstlisting}
(gdb) info registers $rdi $rsi
rdi            0x4006e0	4196064
rsi            0x7fffffffded4	140737488346836
(gdb) x/s 0x4006e0
0x4006e0:	"%d\n"
(gdb) 
\end{lstlisting}

You can use the \texttt{stepi} instruction to step into the call to
\texttt{scanf()}. Take a look around, then return to \texttt{main()} with the
\texttt{return} command.

\begin{lstlisting}
(gdb) stepi 
0x00000000004004e0 in __isoc99_scanf@plt ()
1: x/5i $rip
=> 0x4004e0 <__isoc99_scanf@plt>:	
    jmp    QWORD PTR [rip+0x200b4a]        # 0x601030
   0x4004e6 <__isoc99_scanf@plt+6>:	push   0x3
   0x4004eb <__isoc99_scanf@plt+11>:	jmp    0x4004a0
   0x4004f0:	jmp    QWORD PTR [rip+0x200b02]        # 0x600ff8
   0x4004f6:	xchg   ax,ax
2: x/5xw $rsp
0x7fffffffdec8:	0x0040062d	0x00000000	0xffffdfc0	0x00007fff
0x7fffffffded8:	0x3f318f00
(gdb) where
#0  0x00000000004004e0 in __isoc99_scanf@plt ()
#1  0x000000000040062d in main ()
(gdb) disassemble 
Dump of assembler code for function __isoc99_scanf@plt:
=> 0x00000000004004e0 <+0>:	jmp    QWORD PTR [rip+0x200b4a]        # 0x601030
   0x00000000004004e6 <+6>:	push   0x3
   0x00000000004004eb <+11>:	jmp    0x4004a0
End of assembler dump.
(gdb) return
Make selected stack frame return now? (y or n) y
#0  0x000000000040062d in main ()
(gdb) 
\end{lstlisting}

From here you can exit \texttt{gdb} with the \texttt{quit} command. This
walkthrough has covered enough on \texttt{gdb} to get you started learning about
it on your own. When in doubt, remember to use the \texttt{help} command or
check the man page for \texttt{gdb}.
